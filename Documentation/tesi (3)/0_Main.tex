%%%%%%%%%%%%%%%%%%%%%%%%%%%%%%%%%%%%%%%%%%%%%%%%%%%%%%%%%%%
% PREAMBOLO
%%%%%%%%%%%%%%%%%%%%%%%%%%%%%%%%%%%%%%%%%%%%%%%%%%%%%%%%%%%

% === Impostazione del documento ==========================
\documentclass[12pt,a4paper,twoside,english,italian,hidelinks]{book}


\usepackage{setspace}
\onehalfspace

% === Regolazione dei margini =============================
\addtolength{\oddsidemargin}{30pt}
\addtolength{\evensidemargin}{-30pt}
\usepackage{fancyhdr}
\usepackage{multirow}
\usepackage{multicol}
\usepackage[section]{placeins}

% === Impostazione dei font ===============================

\usepackage[T1]{fontenc}
\usepackage[italian]{babel}
\usepackage{ae}
\usepackage{relsize}
\usepackage{csquotes} 
\usepackage{amsmath}
\usepackage{amsfonts}
\usepackage{mathdots}
\usepackage[colorlinks=true]{hyperref}
\hypersetup{
	bookmarksnumbered=true,
	linkcolor=black,
	citecolor=black,
	pagecolor=black,
	urlcolor=black,
}
\usepackage{verbatim}
\usepackage{alltt}

% === Integrazione delle figure ===========================
\usepackage{graphicx}
\graphicspath{{./imgs/}}
\renewcommand{\figurename}{Fig.}
\usepackage{subfigure}
\usepackage[section]{placeins}
\usepackage[nottoc]{tocbibind}

% === Per gli algoritmi ===================================
\usepackage{algorithmicx}
\usepackage[ruled]{algorithm}
\usepackage{algpseudocode}
\usepackage{listings}
\usepackage{xcolor}

%New colors defined below
\definecolor{codegreen}{rgb}{0,0.6,0}
\definecolor{codegray}{rgb}{0.5,0.5,0.5}
\definecolor{codeorange}{rgb}{0.255,0.112,0.67}
\definecolor{codepurple}{rgb}{0.58,0,0.8}
\definecolor{backcolour}{rgb}{0.250,0.250,0.250}

%Code listing style named "mystyle"
\lstdefinestyle{mystyle}{
  backgroundcolor=\color{white},   commentstyle=\color{codegreen},
  keywordstyle=\color{orange},
  numberstyle=\tiny\color{codegray},
  stringstyle=\color{codepurple},
  basicstyle=\ttfamily\tiny, %or small or footnotestyle
  breakatwhitespace=false,         
  breaklines=true,                 
  captionpos=b,                    
  keepspaces=true,                 
  numbers=left,                    
  numbersep=3pt,                  
  showspaces=false,                
  showstringspaces=false,
  showtabs=false,                  
  tabsize=1
}

%"mystyle" code listing set
\lstset{style=mystyle}

% === Per le tabelle ======================================
\usepackage{tabularx}
\usepackage{multirow}

% === Per la bibliografia multicolonna ====================
\usepackage{etoolbox}
%\usepackage[backend=biber, sorting=none]{biblatex} \addbibresource{bibliografia-tesi.bib}

%\patchcmd{\thebibliography}{\list}{\begin{multicols}{2}\smaller\list}{}{}
%\appto{\endthebibliography}{\end{multicols}}


%%%%%%%%%%%%%%%%%%%%%%%%%%%%%%%%%%%%%%%%%%%%%%%%%%%%%%%%%%%
% TESTO DELLA TESI
%%%%%%%%%%%%%%%%%%%%%%%%%%%%%%%%%%%%%%%%%%%%%%%%%%%%%%%%%%%

\begin{document}
\frontmatter
	% === Frontespizio ====================================
	\pagestyle{empty}
	%%%%%%%%%%%%%%%%%%%%%%%%%%%%%%%%%%%%%%%%%%%%%%%%%%%%%%%%%%%
% Frontespizio
%%%%%%%%%%%%%%%%%%%%%%%%%%%%%%%%%%%%%%%%%%%%%%%%%%%%%%%%%%%

\begin{titlepage}
 \begin{center}
     \includegraphics[width=3.5cm]{imgs/unimol/unimol_color.png}\\
     \vspace{1em}
     {\Large \textsc{Università  degli studi del Molise}}\\
     \vspace{1em}
     {\Large \textsc{Dipartimento di Bioscienze e Territorio}}\\
     \vspace{1em}
      {\Large \textsc{Corso di Laurea in Informatica}}\\
     \vspace{2em}
     {\normalsize Tesi di Laurea in}\\
     \vspace{1em}
     {\Large \textsc{Programmazione Web e Mobile}}\\
     \vspace{3em}
     {\LARGE\textbf{
     TITOLO TITOLO TITOLO\\\vspace{0.4em}
     	TITOLO TITOLO TITOLO
     }}\\
 \end{center}

%spazio tra il titolo e i nomi%
\vskip 1.2cm
  \begin{center}
    \begin{tabular}{c c c c c c c c}
      Relatore & & & & & & & Candidato \\[0.2cm]
      \large{Ing. Francesco Mercaldo} & & & & & & & \large{Andrea D'Aguanno} \\ & & & & & & & {Matricola: 156708}\\[0.4cm] 
       Correlatore & & & & & & & \\[0.2cm]
      \large{Dott.ssa Rosangela Casolare}& & & & & & & \\
    \end{tabular}
  \end{center}

\vskip 1.5cm
\begin{center}
{\normalsize Anno Accademico 2020/2021}
\end{center}
\end{titlepage}

\clearpage{\pagestyle{empty}\cleardoublepage}

	
	%%% inseriamo apgina vuota %%%
	%\clearpage
    %\null
    %\thispagestyle{empty}
    %\clearpage
	%%%%%%%%%%%%%%%%%%%%%%%%%%%%%%%
	%%%%%%%%%%%%%%%%%%%%%%%%%%%%%%%%%%%%%%%%%%%%%%%%%%%%%%%%%%%
% Dedica

\vspace{5em}
\begin{flushright}
  {\normalsize \textit{Ai miei genitori.}}\\
  {\normalsize \textit{Grazie.}}\\
\end{flushright}

\clearpage{\pagestyle{empty}\cleardoublepage}


	% === Indice ==========================================
	\tableofcontents
	\listoffigures
	


	% === Capitoli Tesi ===================================
	\pagestyle{plain}
	%%%%%%%%%%%%%%%%%%%%%%%%%%%%%%%%%%%%%%%%%%%%%%%%%%%%%%%%%%%
\mainmatter %inizia la numeraizione da capo
\chapter{Introduzione}
\label{chap:Introduzione}

%%%%%%%%%%%%%%%%%%%%%%%%%%%%%%%%%%%%%%%%%%%%%%%%%%%%%%%%%%%

\section{Contesto applicativo}
\label{sec:intro1}
Nell'ambito della sicurezza informatica, attraverso il termine malware, abbreviazione di \textit{malicious software}, si vanno ad indicare tutti quei "software malevoli" il cui scopo ultimo è di andare ad interferire con le operazioni svolte da un utente in un dispositivo elettronico. Un malware può agire a diversi livelli, di fatti il termine stesso racchiude diverse categorie di software malevolo: Virus, Ransomware, Trojan, Worm, sono solo alcuni dei possibili tipi di malware. La caratteristica che differenzia un malware da un qualsiasi altro applicativo software risiede nell'intenzionalità dello sviluppatore nel voler realizzare codice malevolo, quindi non rientrano in questo contesto tutti quei software che presentano un bug che mina alla sicurezza dell'applicativo. 
\\È dunque fondamentale studiare i rischi ed i pericoli che riguardano il mondo dell'informatica e più nello specifico quello del mobile computing\footnote{Con il termine mobile computing, si vanno ad indicare tutti quei dispositivi che non sono vincolati dall'utilizzo in una posizione fissa. Rientrano in questa categoria anche i dispositivi mobile come gli smartphone}. Uno dei sistemi operativi più diffusi per i dispositivi mobile è "Android" che grazie alla sua flessibilità permette all'utente di eseguire particolari azioni liberamente all'interno del proprio ecosistema. Android  ad oggi è leader del mercato degli smartphone, ed è quindi il sistema operativo più utilizzato a livello mondiale con 2,5 miliardi di utenti attivi mensilmente nella prima metà del 2019\cite{venturebeat}. La sua diffusione è stata incentivata anche perché non è collegato con un produttore hardware specifico portando il sistema operativo di Google ad avere solo nel 2015 più di quattromila dispositivi diversi che utilizzassero android\cite{4000}. 





%%%%%%%%%%%%%%%%%%%%%%%%%%%%%%%%%%%%%%%%%%%%%%%%%%%%%%%%%%%

\section{Motivazioni e obiettivi}
\label{sec:intro2}
La possibilità di flessibilità dell'ecosistema Android apre le porte ad una maggior intrusione di codice malevolo all'interno degli applicativi, si pensi che oltre alla possibilità di istallare applicazioni dallo store ufficiale \textit{Google Play} è possibile scaricare ed istallare applicazioni attraverso il reperimento di un qualsiasi file \textit{.apk}\footnote{L'estensione apk - Android Package indica un pacchetto contenete tutti i file relativi ad una applicazione Android.} da un generico sito web. Gran parte dei malware Android si diffonde proprio attraverso l'utilizzo di quest'ultima pratica di fatti, scaricando applicazioni da siti terzi non è garantita una preventiva scansione del download per rilevarne la dannosità come avviene attraverso l'utilizzo dello store ufficiale, più nello specifico attraverso l'utilizzo di \textit{Google Play Protect} attivo di default in \textit{Google Play}.
\\Da qui la motivazione che ci ha portato allo sviluppo di questa tesi è legata alla classificazione di queste minacce, partendo da file apk, con l'obiettivo di utilizzare tecniche di machine learning, nello specifico multiple instance learning per la classificazione di applicativi colludenti affetti da malware.


%%%%%%%%%%%%%%%%%%%%%%%%%%%%%%%%%%%%%%%%%%%%%%%%%%%%%%%%%%%

\section{Risultati raggiunti}
\label{sec:intro3}
Dato un file \textit{Apk} si passa alla conversione del codice binario relativo in un file audio, nello specifico un file \textit{Wav}\footnote{L'estensione wav - Waveform Audio File Format è uno standard di formato di file audio, sviluppato da IBM e Microsoft, per memorizzare un flusso di bit audio. \cite{enwiki:1020662001}}, di seguito si passa ad una suddivisione ricavando quindi da un singolo audio più file audio. Da ognuno si è passati all'estrazione di alcune delle caratteristiche che lo compongono. Creando un dataset ad hoc, abbiamo addestrato modelli di multiple instance learning per la classificazione di applicazioni sicure o applicazioni colludenti affette da malware. 


%%%%%%%%%%%%%%%%%%%%%%%%%%%%%%%%%%%%%%%%%%%%%%%%%%%%%%%%%%%

\section{Organizzazione della tesi}
\label{sec:intro4}
La tesi è sviluppata progressivamente nei seguenti capitoli, ai quali va ad aggiungersi il capitolo corrente relativo all'introduzione:


\begin{itemize}
    \item\hyperref[chap:cap2]{\textbf{Capitolo 2}} Android \textbf{TODO: INTEGRARE}
    \item\hyperref[chap:cap3]{\textbf{Capitolo 3}} Sicurezza in ambiente Android \textbf{TODO: INTEGRARE}
    \item\hyperref[chap:cap4]{\textbf{Capitolo 4}} Metodologie \textbf{TODO: INTEGRARE}
    \item\hyperref[chap:cap5]{\textbf{Capitolo 5}} Sperimentazioni\textbf{TODO: INTEGRARE}
    \item\hyperref[chap:conclusioni]{\textbf{Capitolo 5}} Conclusione\textbf{TODO: INTEGRARE}
\end{itemize}


	%%%%%%%%%%%%%%%%%%%%%%%%%%%%%%%%%%%%%%%%%%%%%%%%%%%%%%%%%%%

\chapter{Android}
\label{chap:cap2}

Nell'ultimo decennio, il numero di utenti che utilizza dispositivi mobile è progressivamente aumentato. Questo trend è dovuto ad un aumento dell'affidabilità e delle prestazioni, si hardware che software. Ad oggi gli utenti che utilizzano uno smartphone in tutto il mondo supera i tre miliardi di persone, questo numero è destinato a salire di diverse centinaia di milioni nel costo dei prossimi anni, una stima suggerisce che nel 2023 gli utenti che utilizzeranno questo strumento sarà di circa 4,3 miliardi\cite{numberofuser}, come si può osservare in figura \ref{fig:WordwideNewzoo}.
\begin{figure}[h]
\centering 
\includegraphics[width=0.7\linewidth]{imgs/capitolo2/intro/statistic_id330695_smartphone-users-worldwide-2016-2023.png} 
\caption{ Worldwide; Newzoo; 2016 to 2020} 
\label{fig:WordwideNewzoo} 
\end{figure}

Tra i sistemi operativi più utilizzati c'è il sistema operativo \textit{Android} basti pensare che nel solo anno del 2020, il 71.41\% dei dispositivi mobile, nello specifico tablet e smartphone, venduti aveva come sistema operativo proprio \textit{Andorid}, che insisme ad iOS, hanno coperto il 99.37\% del mercato mobile come si può osservare in figura \ref{fig:ww-sel}. L'ultima versione di questo sistema operativo \textit{Android R - Andorid 11} è stata rilasciata nel settembre 2020, ma la successiva versione Beta di \textit{andorid 12} è prossima al rilascio. 
\begin{figure}[h]
\centering 
\includegraphics[width=0.7\linewidth]{imgs/capitolo2/intro/StatCounter-os_combined-ww-monthly-202001-202012-bar.png}
\caption{Mobile \& Tablet Operating System Market Share Worldwide} 
\label{fig:ww-sel} 
\end{figure}
\FloatBarrier
 La diffusione del sistema operativo \textit{Andorid} è avvenuta per mezzo di smartphone e tablet, ma sono state sviluppate soluzioni ottimizzare per dispositivi specifici come \textit{Wear Os} per smartwatch, \textit{Android TV} per smart tv, \textit{Android Auto} per l'integrazione tra smartphone ed auto, che ne aumentano ulteriormente il bacino di utenza. 
\\
Nei prossimi paragrafi esporremo l'architettura del sistema operativo \textit{Android} ed i principali componenti che formano un applicativo Android. Infine descriveremo i file apk.  

%%%%%%%%%%%%%%%%%%%%%%%%%%%%%%%%%%%%%%%%%%%%%%%%%%%%%%%%%%%

\section{Il sistema operativo}
\label{sec:Il sistema operativo}
Come già detto in precedenza Android è un sistema operativo per dispositivi mobile, sviluppato dall'azienda \textit{Android, Inc.} che fu poi acquistata nel 2005 dalla statunitense Google che lo ha poi diffuso nel 2008. Il modello di sviluppo è open source, questo consente a chiunque sia interessato di progettare e sviluppare componenti software ad esso dedicati, anche grazie alle librerie e alla documentazione fornita dal produttore. Inoltre essendo distribuito con licenza \textit{Apache 2.0} consente a chiunque di modificare e distribuire il codice sorgente. Le applicazioni sviluppate per dispositivi Android sono scritte in linguaggio Java o Kotlin, linguaggi molto diffusi che consentono di prendere parte alla progettazione di applicazioni a molti sviluppatori, questo per far si che siano sempre molte ed aggiornate le funzionalità di un dispositivo. 

\subsection{L'architettura}
L'architettura andorid può essere suddivisa in più livelli come in figura\ref{fig:andoridStack}. 
\begin{figure}[h]
\centering 
\includegraphics[width=0.4\linewidth]{imgs/capitolo2/os/android-stack_2x.png}
\caption{Android architecture} 
\label{fig:andoridStack} 
\end{figure}

\begin{itemize}
\item Nel livello più basso troviamo il \textbf{kernel linux} al quale Android si appoggia per i servizi del sistema centrale quali, sicurezza, gestione della memoria. Inoltre presenta dei driver specifici per la gestione dell'hardware. Il Kernel funge quindi da strato di comunicazione tra hardware e software.
\\Driver Binder(IPC): Questo sistema consente al sistema operativo la comunicazione tra i processi gestendo il passaggio dei dati tra le applicazioni che in Android sono eseguite ognuna in un processo differente;
\\Driver Shared Memory: consente la creazione, la mappatura ed il controllo della protezione della memoria condivisa tra i diversi processi. 

Nonostante la scelta di utilizzare un kernel linux, per ottenre affidabilita e garantire sicurezza, Andoroid è considerato come una distribuzione embedded Linux, sviluppata appositamente per ottimizzare al meglio le risorse, spesso esigue, di un dispositivo mobile e non è quindi una distribuzione Unix-like, ovvero non fa parte di tutti quei sistemi operativi simili discendenti dal sistema operativo Unix. 
\item Salendo di livello troviamo una serie di \textbf{libreirie native} sviluppate in C / C++ e libreire sviluppate in Java: 
\\\textbf{Media Framework}: è la componente in grado di gestire diversi CODEC per i vari formati di acquisizione e riproduzione di audio e video. È basata sulla libreira open surce OpenCORE. Supporta anche formati immagini come png e jpg. 
\\\textbf{Surface Manager}: Gestisce l’accesso alle funzionalità del display e coordina le diverse finestre che le applicazioni vogliono visualizzare sullo schermo, permettendo la visualizzazione contemporanea di grafica 2D e 3D dalle diverse applicazioni.
\\\textbf{OpenGL ES}: Attraverso le API implementate in questa libreria si possono gestire le funzionalità 2D e 3D e l'utilizzo di un eventuale acceleratore grafico. 
\\\textbf{SQLite}: Libreria che implementa un database relazionale.
\\\textbf{WebKit}: Browser utilizzato da Android per la visualizzazione di risosrse web. 
\\\textbf{Libc (System C library)}: Implementazione della libreria standard C system (libc), per i dispositivi basati su Linux embedded ad eempio Android.
\\\textbf{Secure Socket Layer (SSL)}: Si occupa della sicurezza attraverso la gestione dei Secure Socket Layer. Sono protocolli crittografici che permettono una comunicazione sicura e una integrità dei dati su reti TCP/IP. I SSL cifrano la comunicazione dalla sorgente alla destinazione a livello di trasporto.
\item \textbf{L'Android runtime (ART)} è un runtime system software che ha sostituito la Dalvik virtual machine (DVM), quest'ultima infatti si occupava di compilare il codice ad ogni esecuzione di un'applicazione incidendo sulle prestazioni del dispositivo. Con ART invece la compilazione del codice avviene durante l'istallazione dell'applicazione, questo comporta tempi più lunghi per la procedura di istallazione ma una volta terminata, l'esecuzione dell'applicativo non necessita di compilazione. Per mantenere la compatibilità con le versioni precedenti, il bytecode utilizzato da ART per la compilazione è lo stesso utilizzato da DVM ovvero quello fornito dal file \textit{.dex}\footnote{Le applicazioni Android sono comunemente scritti in Java e successivamente compilati in bytecode per la macchina virtuale Java, che viene quindi tradotto in bytecode Dalvik e archiviato in file .dex (Dalvik EXecutable ) oppure .odex (Optimized Dalvik EXecutable)}
\item \textbf{Java API Framework} Consiste di Api e componenti sviluppate in Java per l'esecuzione di precise funzionalità di un'applicazione Android. 
\\\textbf{Activity Manager} Responsabile dell'organizzazione della schermate in uno stack. Rappresenta lo strumento attraverso il quale avviene l'interazione con l'utente. 
\\Attraverso il \textbf{{Content Provvider}} si vanno a gestire gli accessi ai dati archiviati dalla stessa applicazione o da terze, fornisce quindi meccanismi per la condivisione di dati tra applicazioni. 
\\\textbf{Notification Manager} implementa i metodi per inviare una notifica al dispositivo ed avvisare l'utente che qualcosa è successo in background. 
\\Con l'utilizzo del \textbf{View System} si può renderizzare la GUI (graphical user interface) attraverso l'utilizzo di pulsanti, tabelle, aree di testo e così via in modo da poter gestire gli eventi associati ad ogni elemento. Il codice è contenuto in un file di marckup .xml\footnote{XML - eXtensible Markup Language, è un metalinguaggio per la definizione di linguaggi di markup dove oltre all'utilizzo di tag prestabiliti è possibile definirne dei propri a seconda del contesto applicativo.}.
\item Il primo livello, \textbf{System Apps}, contiene tutte le applicazioni native e istallate all'interno del dispositivo Android. 
\end{itemize}


%%%%%%%%%%%%%%%%%%%%%%%%%%%%%%%%%%%%%%%%%%%%%%%%%%%%%%%%%%%

\section{Le applicazioni Android}
\label{sub:aplicationAndroid}
In questo paragrafo analizzeremo la struttura di una applicazione Andorid ed i componenti da cui è composta.
\\Lo sviluppo delle applicazioni Android può avvenire attraverso l'Ide\footnote{IDE - integrated development environment è un ambiente di sviluppo, un software che, in fase di programmazione, supporta i programmatori nello sviluppo e debugging del codice sorgente di un programma \cite{itwiki:118335792} } \href{https://developer.android.com/studio/index.html}{\textit{Android Studio}} basato su JetBrains IDEA e rappresenta l'ide principale per lo sviluppo Android di Google. I linguaggi di programmazioni utilizzati per sviluppare un applicativo Andori dcome detto anche in precedenza sono \href{https://www.java.com/it/}{Java} e \href{https://kotlinlang.org/}{Kotlin}, quest'ultimo inoltre sta diventando sempre più consigliato e diffuso in ambito Android. Lo sviluppo avviene utilizzando il kit di supporto software Android (SDK) che comprende documentazione dei metodi, un debugger, librerie software ed un emulatore, in Android studio è possibile gestire ed utilizzate queste componenti da "SDK Manager". L'attività di \textit{run} del codice può avvenire sia su un dispositivo fisico collegato tramite usb e dopo aver attivato la modalità "Debug Usb" oppure scaricando ed istallando un immagine di Android nel "AVD Manager - Android virtual devices" 

\subsection{Struttura}
La struttura che compone un progetto Android può essere suddivisa in tre moduli. Tutto il progetto sarà contenuto in una cartella \textit{app}. Al interno troveremo ulteriori directory che conterranno il codice e le risorse dell'applicativo: 
\begin{itemize}
    \item \textbf{manifest}: Le applicazioni Andorid dispongono di un file AndroidManifest.xml, in questo file vanno definite tutti i componenti sviluppati, la versione minima di API necessarie per eseguire l'applicazione, i permessi di cui necessita l'applicazione ed infine elenca le librerie esterne utilizzate nel codice.  
    \item \textbf{java}: All'interno della directory java è contenuto il codice sorgente delle componenti e della logica che va a sviluppare l'applicativo, inoltre presenta anche dei file si test
    \item \textbf{res}: Le risorse che non sono codice, come imamgini, layout xml, stringhe della GUI sono contenute all'interno di subdirectory di res, nello specifico troviamo la directory \textit{drawble} ceh conterrà le immagini, la directory \textit{layout} al suo definisce tutti i file di layout.xml associati ad ogni songolo activity e/o fragment creato. Infine in \textit{values} sono contenuti vari marcatori xml utili per la definizioni di stringhe, dimensioni, colori e stili.   
\end{itemize}
Il secondo modulo, la sezione \textit{Grandle Scripts}, comprende tutti i file utilizzati per la build del progetto
\subsection{Componenti}
Le componenti principali che compongono per l'appunto un applicativo sono quatto: Activity, Service, Content Provvider, Broadcast Reciver e Intent. 

   

    \subsubsection{Activity} Un \textbf{Activity} rappresenta l'Interfaccia utente, ogni schermata è un activity, all'interno della quale possono alternarsi dei \textit{fragment}. All'interno del codice vie effettuato l'override del metoso "onCreate" in cui viene specificato il suo layout attraverso setContentView(R.layout.activity). Le acticity in Andorid seguono un ciclo di vita ben definito come si può osservare nella figura \ref{fig:lifecicle} tratta dalla \href{https://developer.android.com/guide/components/activities/activity-lifecycle}{documentazione ufficiale.}
        \begin{figure}[h]
        \centering
        \includegraphics[width=0.35\textwidth]{imgs/capitolo2/Applicazioni/activity_lifecycle.png}
        \caption{Android lifecicle}
        \label{fig:lifecicle}
        \end{figure}
        \FloatBarrier %  evitare che i float appaiano oltre un certo punto nel tuo documento %
    Nello specifico il metodo \textit{onCreate()} viene invocato per stanziare e definire l'activity, questo viene avviato una sola volta per ciclo di vita. Successivamente quando l'activity è stata creata si invoca il metodo \textit{onStart()} che rende l'activity visibile all'utente, in modo che passi in primo piano e diventi interattiva, invocando il metodo \textit{onResume()} l'activity si pone in "ascolto" di un evento da parte dell'utente, qui definiamo l'interattività con l'user. Rimarrà in questo stato finché non accade un evento che richiede un cambio di activity (ricezione chiamata, navigazione tra activity...), il metodo invocato a questo punto sarà \textit{onPause()} che salva lo stato corrente dell'activity e dunque non sarà più possibile interagire con essa, il passo successivo comprende l'invocazione di \textit{onStop()} metodo che non rende più visibile l'activity. A questo punto l'utente può tornare all'activity ed il sistema richiamerà onResume() oppure si invocherà \textit{onDestroy()} che provvede alla pulizia e cessazione di un'activity. 

    \subsubsection{Service} Un importante componente sono i \textbf{Service}. Questi svolge operazioni in background quindi anche al di fuori dell'utilizzo diretto dell'utente dell'applicazione, difatti non ha un'interfaccia grafica. È inoltre possibile eseguire comunicazione tra processi tramite IPC. I service vanno anchessi definiti nel file AndoridManifest.xml.

    \subsubsection{Conten Provvider} Come detto in precedenza un altro componente principale è il \textbf{Content Provider} che si occupa della gestione della condivisione in memoria dei dati salvati in database , su file oppure in rete tra le applicazioni. Risulta fondamentale per il controllo delle autorizzazioni per l'accesso ai dati.
    
    \subsubsection{Broadcast reciver} Il \textbf{Broadcast reciver} è un componente che dato un messaggio a livello di sistema, in broadcast, consente la reazione all'evento. 
    
    \subsubsection{Intent} L'ultimo componente in esame è \textbf{l'intent}, questo consente di notificare l'intenzionalità di una applicazione nel voler attivare della stessa o di un'altra applicazione in modo da poterne richiedere la funzionalità attraverso delle invocazioni anche di servizi, di broadcast reciver. Il concetto di riuso delle componenti è fondamentale per aver un codice comprensibile e funzionale. I tipi di intent sono: \textit{intent espliciti} che specificano il componente da avviare ed \textit{intent impliciti} ovvero colore che dichiarano un'azione da da eseguire \footnote{Ad esempio un intent implicito può essere la richiesta di scelta di quale applicazione utilizzare per la visualizzazione di un particolare file.} Attraverso l'utilizzo di intent è possibile anche l'invio di parametri extra per lo scambio di informazioni tra oggetti. 


\subsection{Gestione dei dati}
Android permette il salvataggio dei dati dell'applicazioni in diverse modalità: \textbf{Interal storage} ovvero lo spazio riservato esclusivamente all'uso dell'applicazione, in quest'area di memoria le altre applicazioni non possono accedere, per farlo bisogna definire un meccanismo sicuro attraverso il conrent provider di cui abbiamo parlato precedentemente. Attraverso \textbf{l'External storage} le informazioni salvate sono invece accessibili a tutte le applicazioni del dispositivo, di fatti rappresenta uno spazio condiviso tra le varie applicazioni. Se si vuole utilizzare questo metodo si salvataggio è buona norma controllare attraverso il file manifest che il dispositivo supporti l'estensione della memoria, inoltre se i dati sono sensibili bisogna provvedere all'utilizzo cifrati per nascondere le informazioni. È possibile poi usufruire delle \textbf{Shared-preferences} ovvero il salvataggio di piccole raccolte di dati key-value facilmente definibili ed accessibili. Possono essere sia private che condivisibili. Di seguito è mostrato un esempio di utilizzo di shared preferences per il salvataggio di un orario tramite valori passati come parametri della funzione "doSave" \ref{lst:shared}. 
\begin{lstlisting}[language=Java, caption=Shared Preference example, label= lst:shared]
public void doSave(int hour, int minutes) {
        SharedPreferences sharedPreferences = CONTEXT.getSharedPreferences("DispensaSetting", Context.MODE_PRIVATE);
        SharedPreferences.Editor editor = sharedPreferences.edit();
        editor.putInt("hourpreferences", hour);
        editor.putInt("minutepreferences", minutes);
        editor.apply();
}
\end{lstlisting}
Un ulteriore possibilità per il salvataggio delle informazioni è l'utilizzo di \textbf{database}. Android mette a disposizione un database relazionale open source \textit{SQLite}, è stato scelto per la sua ottimizzazione nel consumo di risorse spaziali. SQLite legge e scrive direttamente da e su file su disco. Android utilizza uno schema ben definito per la memorizzazione di file su database, per ogni applicazione che ne fa uso esiste una cartella dedicata allo scopo di contenere i file relativi il database "/data/data/packagename/databases". Ancora una volta per gestire al meglio ed in sicurezza l'accesso alle informazioni in un database, un ruolo fondamentale è ricoperto da content provider. Utilizzando la libreira \href{https://developer.android.com/training/data-storage/room}{ROOM} si può usufruire di un livello di astrazione su SQLite che consentono di avere una controllo in fase di compilazione sulle query SQL ma soprattutto una gestione semplificata del percorso di migrazione del database. L'api ROOM ha tre componenti: \textit{database} classe astratta che rappresenta il database e funge da punto di accesso per la connessione ai dati, \textit{entity} ovvero la classe che rappresenta una tabella del database, \textit{dao} interfaccia con i metodi per effettuare le operazioni di CRUD\footnote{CRUD - create, read, update, and delete sono le quattro operazioni di base della memorizzazione in un database.}. Di seguito un esempio di una interfaccia contente i metodi dao \ref{lst:daoalg}.
\begin{lstlisting}[language=Java, caption=Dao example, label = lst:daoalg]
@Dao
public interface ProdottoDao {
    @Insert
    public Long insertProdotto(ProdottoEntity prodottoEntity);
    @Update
    public void updateProdotto(ProdottoEntity prodottoEntity);
    @Delete
    public void deleteProdotto(ProdottoEntity prodottoEntity);
    @Query("SELECT * FROM prodottoentity")
    public List<ProdottoEntity> findAll();
    @Query("SELECT * FROM prodottoentity WHERE category LIKE :callby")
    public List<ProdottoEntity> findAllByCategory(String callby);
    }
\end{lstlisting}
Infine un'ulteriore possibilità è quella che sfrutta la connessione internet per eseguire il salvataggio dei dati in cloud. 

%%%%%%%%%%%%%%%%%%%%%%%%%%%%%%%%%%%%%%%%%%%%%%%%%%%%%%%%%%%

\section{Il package APK}
\label{sub:apk}
Il .apk è l'estensione che indica che stiamo lavorando con una cartella compressa che racchiude tutti i file di un specifica applicazione Android.Sono una variante del formato .JAR\footnote{Un file con estensione JAR - Java Archive indica un archivio dati compresso usato per distribuire raccolte di classi Java.}. In figura \ref{fig:apkPack} possiamo osservare la composizione di un package apk. Più nello specifico, un file APK è un archivio che contiene i seguenti file: 
\textit{AndroidManifest.xml}, \textit{classes.dex} e \textit{resurces.arsc} inoltre comprende anche le cartelle \textit{META-INF} e \textit{res}
        \begin{figure}[h]
        \centering
        \includegraphics[width=0.6\textwidth]{imgs/capitolo2/Applicazioni/apk ins.png}
        \caption{Apk package}
        \label{fig:apkPack}
        \end{figure}
    \FloatBarrier %  evitare che i float appaiano oltre un certo punto nel tuo documento %
\subsubsection{Classes.dex}    
Questo file è composto dall'insieme delle classi che compongono la logica dell'applicativo android. Il risultato sono classi compilate dell'Android SDK per essere eseguite dalla macchina virtuale Android RunTime (ART). Trovandoci in ambiente si sviluppo java, si ha che ogni classe sarà compilata in un file .class, questo passaggio avviene anche in Android a cui però segue la conversione di tutti i file .class in un unico file .dex. Dunque questo file non conterrà il bytecode\footnote{Il bytecode è un linguaggio intermedio che si posiziona tra il linguaggio macchina e il linguaggio di programmazione. Viene adoperato per descrivere le operazioni che costituiscono un programma.} java, ma il bytecode DEX che è stato introdotto appositamente per il sistema android in quando, tra le altre, va ad ottimizzare l'uso della memoria.
\subsubsection{Res e resurces.arsc}
Nell'package apk troviamo una cartella \textit{res}, questa contiene tutte le risorse dell'applicativo come le immagini ed il layout dei file xml. Mentre \textit{resurces.arsc} è un file di risorse questa volta però compilate. 
\subsubsection{META-INF}
Questa cartella contiene le informazioni del manifest ed altri metadati\footnote{Sistema di dati il cui scopo è la descrizione di altri dati, compresi gli archivi elettronici.}. Nello specifico il file \textit{MANIFEST.MF} contiene tutte quelle informazioni utilizzate da ART in fase di run-time, come ad esempio qual è la classe principale e quali sono le politiche di sicurezza. 


%%%%%%%%%%%%%%%%%%%%%%%%%%%%%%%%%%%%%%%%%%%%%%%%%%%%%%%%%%%


%%%%%%%%%%%%%%%%%%%%%%%%%%%%%%%%%%%%%%%%%%%%%%%%%%%%%%%%%%%



%%%%%%%%%%%%%%%%%%%%%%%%%%%%%%%%%%%%%%%%%%%%%%



	\chapter{Sicurezza in ambiente Android}
\label{chap:cap3}
Android, come detto, è il leader mondiale del mercato mobile, questo ha portato all'aumento dello sviluppo di diversi tipi di applicazioni aumentando di conseguenza anche il numero degli attacchi attraverso l'utilizzo di applicazioni affette da malware. Un rapporto di Avira descrive come nella prima metà del 2020 si potessero contare quasi 2 milioni di applicazioni android affette da malware\cite{newMalwareAvira} figura \ref{fig:avira}, nell'anno precedente le rilevazioni effettuate da G DATA mostravano lo stesso trend\cite{newMalware} come si può osservare in figura \ref{fig:GSATA}. 
     \begin{figure}[h]
        \centering
        \includegraphics[width=0.6\textwidth]{imgs/capitolo3/G_DATA-Infographic-MMR-HJ1-2019-New_Android_Malware-monthly-EN-Logo.jpg}
        \caption{Malware detection 2019 - G DATA}
        \label{fig:GSATA}
\end{figure}
\FloatBarrier %  evitare che i float appaiano oltre un certo punto nel tuo documento %
\begin{figure}[h]
        \centering
        \includegraphics[width=0.6\textwidth]{imgs/capitolo3/avira.png}
        \caption{malware detection 2020 - Avira}
        \label{fig:avira}
        \end{figure}
\FloatBarrier %  evitare che i float appaiano oltre un certo punto nel tuo documento %
Il malware inoltre è stato la categoria di attacco che più è stata rilevata come minaccia in ambiente android, quasi $3/4$ delle rilevazioni riguardavano appunto un malware\cite{newMalwareAvira}.
Questo è stato reso possibile dalla frammentazione del mercato. Essendo un sistema operativo in continua evoluzione e diffusione diventa sempre più complesso riuscire a garantire la sicurezza di tutte le versioni in circolazione. Di fatti un problema per la sicurezza è legato alla versione android istallata sul proprio device. Nel 2017 si stimava ci fossero circa 1 miliardo di dispositivi android che non fossero aggiornati o non avrebbero ricevuto aggiornamenti, rendendo i dispositivi sempre più obsoleti e vulnerabili\cite{oneMilion}. 

In figura \ref{fig:os version}, nell'intervallo che va dal Gennaio 2019 al Gennaio 2021 possiamo infatti osservare come le due versioni più diffuse siano Android 9 (2017) e Android 8 (2018) 

    \begin{figure}[h]
        \centering
        \includegraphics[width=0.6\textwidth]{imgs/capitolo3/StatCounter-android_version-ww-monthly-201901-202101-bar.png}
        \caption{Android OS Version - GlobalStats statcounter}
        \label{fig:os version}
        \end{figure}
        \FloatBarrier %  evitare che i float appaiano oltre un certo punto nel tuo documento %


Il rischio rimane quindi molto alto e da non sottovalutare. 
Nel prossimo paragrafo vedremo cosa si intende per malware. 
\section{Malware}
un Malicius software abbreviato, Malware, è un termine che va ad indicare tutti quei programmi che mettono a rischio un sistema informatico. La maggior diffusione avviene attraverso internet e più nello specifico attraverso le e-mail. In ambiente mobile però le app dannose possono nascondersi anche all'interno di applicazioni che all'apparenza sembrano non rappresentare una minaccia, questo avviene soprattutto se ci si affida per il download a store non ufficiali. I tipi di malware più diffusi sono: 
\begin{itemize}
    \item \textit{Virus}: programmi presenti in applicazini che una volta eseguite diffondono il codice malevole ad altri programmi del sistema;
    \item \textit{Trojan}: codice che solitamente è nascosto in applicazioni che riusltano utili all'utente ma che una volta istallate consentono agli attaccanti di ottenere l'accesso al dispositivo; 
    \item \textit{Ransomware}: impediscono all'utente di accedere al proprio dispositivo cifrando i suoi file, spesso sono seguiti da una richiesta di riscatto per riottenere l'accesso al dispositivo; 
    \item \textit{Worm} si diffondono nei dispositivi di una rete danneggiandoli mediante la distruzione di dati e file;
\end{itemize}

Da un'indaggine condotta da AV-Test, i trojan sono risultati il mezzo preferito dai criminali informatici per introdurre codice malevolo rappresentando il 93.93\%  di tutti gli attacchi di malware sui sistemi Android. Il ransomware si è classificato al secondo posto, con il 2,47\% \cite{trojan}.

\subsection{Colluding}
Le applicazioni colludono perché si scambiano informazioni attraverso shared\_preferences, broadcast, external\_storage, quindi ogni set ha due applicazioni che sono in comunicazione tra loro. Le app di get salvano l'informazione mentre quelle di put ricevono l'informazione e la vanno a modificare.  
\textbf{TODO: INTEGRARE}


\textbf{Francesco: parliamo di colluding, dicendo gli attacchi che dopo andremo a testare, il fattoc he la tua metolodgia applica il multiple instance learning per rilevare applicazioni collusive deve essere chiaro sin dall' inizio della tesi}
	\chapter{La metodologia}
\label{chap:cap4}
\textbf{Francesco: non classifichi file audio...ma applicazioni Android attraverso file audio...}
In questo capitolo esporremo la metodologia utilizzata per la classificazione delle applicazioni Android, attraverso file audio. Nello specifico partiremo dai software e dai tipi di dati utilizzati per poi muoverci verso l'estrazione delle feature, la generazione dei dataset e la classificazione attraverso una particolare tecnica di machine learning chiamata multiple instance learning (MIL). 
\begin{figure}[h]
\centering
    \includegraphics[width=0.9\linewidth]{imgs/capitolo4/engall.png} 
    \caption{Methodology of development  }
    \label{fig:all}
\end{figure}
\FloatBarrier
\section{I software e i dataset}
In questo paragrafo esporremo dapprima tutti i software utilizzati nell'analisi e una breve panoramica sulle caratteristiche principali dei dataset utilizzati.  
\subsection{WEKA}
\label{par:weka}
Acronimo di "Waikato Environment for Knowledge Analysis", è un software open source per il machine learning. Partendo da un dataset\footnote{Collezione di dati organizzati, la grandezza è data dal numero di righe.} è possibile applicarvi dei metodi di apprendimento automatico e di analizzarne il risultato è inoltre possibile attraverso l'utilizzo di questi metodi avere una previsione su nuovi set di dati. Per poter utilizzare gli algoritmi di classificazione del multiple instance learning bisogna importare i relativi package, attraverso il tool "Package Manager" già presente di default nella schermata iniziale di weka. Figura \ref{fig:mil pckg}.
\begin{figure}[h]
\centering
    \includegraphics[width=0.9\linewidth]{imgs/capitolo4/packmil.png} 
    \caption{Input Mil package in weka}
    \label{fig:mil pckg}
\end{figure}
\FloatBarrier
Una particolarità di questo software è l'utilizzo di dataset .arff 


\subsection{dataset.csv e dataset.arff}
\label{par:dataset}
I dataset utilizzati nel progetto sono dataset con estensione \textbf{.csv - Comma Separated Values}, questo è un formato di file di testo in cui ogni riga rappresenta un record della tabella. Ogni colonna invece rappresenta dei valori associati ad ogni record. Le colonne sono separate da virgole, da qui il nome. In figura \ref{Fig:Datacsv} è possibile osservarne un esempio. 
\vspace{1em}
\newline
Il secondo tipo di dataset che abbiamo utilizzato sono file dati con estensione \textbf{.arff - Attribute Relationship File Format} come suggerisce il nome, questo formato di file organizza i dati seguendo una logica relazionale.
La formattazione del dataset utilizzato è stata realizzata in ottica di una classificazione attraverso algoritmi di multiple instance learning, dunque si è reso necessario dover organizzare i dati in bag. L'inizializzazione del file, per una classificazione MIL\footnote{Mil - Multiple instance learning} può essere suddiviso in cinque componenti\cite{wekaDoc}:  
\begin{enumerate}
        \item Nella prima va sempre definita la relazione che lega i dati attraverso l'attributo \textbf{@relation} ed un nome che descriva quello che vogliamo predire.  
        
        \item Successivamente andranno inseriti gli identificativi delle bag, ovvero utilizzando \textbf{@attribute bag\_id \{...\}} si vanno ad inserire nelle parentesi graffe la lista di tutti gli identificativi delle istanze della bag. Ogni identificativo va separato dall'altro tramite tramite l'utilizzo di una virgola. 
        
        \item Dopo si definiranno gli attributi della bag, per farlo si utilizza \textbf{@attribute bag relational} per definire l'inizio della bag ed \textbf{@end bag} per definire la fine della bag. All'interno, tra i due attributi, vanno specificati gli attributi che compongono le istanze di una bag, ovvero le caratteristiche dei dati. Per farlo si utilizza ancora una volta l'identificativo \textbf{@attribute nome\_caratteristica tipo\_caratteristica}. Il tipo di caratteristica può essere \textit{numeric} se il valore del dato è un numero intero, altrimenti \textit{real} se il tipo di attributo è un numero reale altrimenti. Se invece il tipo di attributo è una stringa scriveremo i valori ammissibili presenti tra parentesi graffe es. {\{yes, no\}} nel caso di un attributo booleano.
        
        \item A questo punto dobbiamo definire la classe che rappresenta una istanza. Per farlo inseriremo i valori tra parentesi graffe definendo l'attributo class come, \textbf{@attribute class \{calss1, class2\}}
        
        \item Infine a capo della key \textbf{@data} inseriremo il dataset correttamente formattato nel seguente modo: iesima bag\_id + ',' poi dovremmo definire tutte le istanze della bag. Una bag si definisca all'interno delle virgolette "...". Inseriremo all'interno tutte le istanze ognuna della quali sarà separata dal carattere speciale '$\backslash$n'. A loro volta ogni istanza è rappresentata dai diversi attributi, tanti quanti ne abbiamo definiti in precedenza, ogni attributo d'istanza è separato dall'altro tramite ','. Infine dopo la chiusura delle virgolette inseriamo la virgola e va definita la classe della bag. Ogni riga ha quindi la seguente formattazione:
        \\\footnotesize {
        bag\_id , " attr1Ist1, attr2Ist1, attr3Ist1 $\backslash$n attr1Ist2, attr2Ist2, attr3Ist2 $\backslash$n attr1Ist3, attr2Ist3, attr3Ist3 " , classe  }
    \end{enumerate}
\begin{figure}[h]
   \begin{minipage}{0.48\textwidth}
     \centering
     \includegraphics[width=0.85\linewidth]{imgs/capitolo4/ARFF.png}
     \caption{ARFF dataset for MIL}
     \label{Fig:Dataarff}
   \end{minipage}\hfill
   \begin{minipage}{0.48\textwidth}
     \centering
     \includegraphics[width=0.85\linewidth]{imgs/capitolo4/csv.png}
     \caption{CSV file}
     \label{Fig:Datacsv}
   \end{minipage}
\end{figure}
\FloatBarrier 
\subsection{Dataset di applicativi android}
\label{par:datasetApk}
Siamo partiti da due dataset di applicazioni apk. Il primo conteneva un set di 200 applicazioni android non affette da malware che abbiamo definito come dataset "trusted", il secondo dataset invece conteneva 241 coppie di apk affette da malware abbiamo definito questo dataset come "Acid". Quest'ultimo dataset come detto comprendeva 241 set ognuno dei quali includeva due package APK. Nello specifico queste due applicazioni, affette da malware, comunicano tra loro attraverso lo scambio nascosto di informazioni e vengono quindi dette \textit{colludenti}. Di fatti una applicazione è di tipo "PUT" e l'altra è di tipo "GET". Per un totale di 482 applicazioni colludenti, si hanno 241 applicazioni di tipo "GET" e 241 di tipo "PUT".

\section{Elaborazione dei file audio}
In questo paragrafo descriveremo l'elaborazione dei file audio, ovvero la loro generazione/conversione e la successiva suddivisione.

\subsection{Generazione dei file audio}
\label{par:gen}
Lo script realizzato per la conversione di un applicativo android in un file audio, una volta inserito nella directory che contiene i due dataset di apk, va dapprima ad individuare tra i tutti i file che compongono un dataset i soli package .apk, successivamente va a decomprimere ogni pacchetto e ad estrarre da ognuno il file classes.dex. Questo file viene elaborato e  convertito in un file audio con estensione WAV. L'output è salvato in una cartella creata ad hoc per raccogliere tutti i file audio generati,  suddividendo i file audio generati da applicativi "trusted" da quelli di tipo "Acid", creando una directory di output differente a seconda del tipo. 

\subsection{Splitting dei file audio}
Il secondo script sviluppato chiede in input la durata dello split in secondi, la durata dev’essere maggiore di 0.
Controllando che non sia stata creata in precedenza, va ad inizializzare una cartella "Splitted" tramite una funzione “createNewDirectory” [Listing:\ref{lst:newDir}].
\begin{lstlisting}[language=Python, caption=Create new directory function, label = lst:newDir]
def createNewDirectory(path=os.getcwd(), nameNewDirectory=""):
    pathNewDirectory = path + "\\" + nameNewDirectory
    if not os.path.exists(pathNewDirectory):
        try:
            os.mkdir(pathNewDirectory)
            print("Directory created successfully")
            return pathNewDirectory
        except OSError as error:
            print("Directory can not be created")
    else:
        return pathNewDirectory
\end{lstlisting}

Proseguendo va ad iterare su tutti i file audio WAV presenti nelle cartelle "trusted" ed "Acid" generate precedentemente dalla conversione delle applicazioni. [Listing:\ref{lst:audioOrig}]

\begin{lstlisting}[language=Python, caption=Iteration over whole audios, label = lst:audioOrig]
for wav_file in os.listdir(unsplit_audio_folder):
    if wav_file.endswith(".wav"):
        wavSplittedDirectory = createNewDirectory(Path_SubDirectory, str(wav_file))
        pathToOriginalWav = unsplit_audio_folder + "\\" + str(wav_file) 
        splittingWav(pathToOriginalWav, wavSplittedDirectory)
\end{lstlisting}


Per ogni file WAV trovato si va a creare uan cartella in "\textit{Splitted\textbackslash nameFileAudio}" dove il nome della directory è dato dal file .wav da splittare. Questa nuova cartella sarà poi popolata con i file splittati risultanti. In questo modo ogni file .wav avrà una sua directory che conterrà gli split di output.\\
Lo split avviene in una funzione “splittingWav” che va a calcolare il numero degli split in cui suddividere il file audio originale. Il calcolo avviene tramite una divisione per eccesso, in questo modo si includono anche suddivisioni più piccole del numero dato in input. Successivamente iterando sugli intervalli crea la suddivisione, partendo di volta in volta dal file della durata intera, come si può osservare in figura \ref{fig:split1} e \ref{fig:split2}.
\begin{figure}[h]
   \begin{minipage}{0.48\textwidth}
     \centering
     \includegraphics[width=0.85\linewidth]{imgs/capitolo4/suddivisione.png}
     \caption{first split}
         \label{fig:split2}
   \end{minipage}\hfill
   \begin{minipage}{0.48\textwidth}
     \centering
     \includegraphics[width=0.85\linewidth]{imgs/capitolo4/suddivisione2.png}
     \caption{second split}
     \label{fig:split1}
   \end{minipage}
   
\end{figure}
\FloatBarrier 
La libreria utilizzata per effettuare lo splitting dei dati è \href{https://github.com/jiaaro/pydub#installation}{"pydub"}\cite{pydub}. Nel codice le variabili @t\_start e @t\_stop rappresentano per ogni iterata il punto d'inizio e quello di fine dell'i-esimo split del file. Il nome dei file splittati corrispondono all’intervallo dell’i-esimo split (in millisecondi). 

\begin{lstlisting}[language=Python, caption=Splitting wav function, label = lst:splitFun]
def splittingWav(pathToOriginalWav, pathToSplittedWav):
    #Load
    originalAudio = AudioSegment.from_wav(pathToOriginalWav)
    numberOfSplit = math.ceil(originalAudio.duration_seconds / splittingduration)
    t_start = 0
    
    for i in range(numberOfSplit):
        t_stop = (t_start + (splittingduration * 1000))  
        splittedAudio = originalAudio[t_start:t_stop]
        splittedAudio.export(out_f=pathToSplittedWav + "\\" + str(t_start) + " - " + str(t_stop) + '.wav',format="wav")         
        t_start = t_stop
\end{lstlisting}

\section{Estrazione delle features}
\label{par:feat}
Al fine di poter effettuare una classificazione abbiamo bisogno di attributi che possano rappresentare un dato oggetto, nel nostro caso un audio digitale. Il suono in natura è un segnale continuo, dunque per essere memorizzato dev'essere campionato ottenendo in questo modo un segnale digitale rappresentato da valori numerici che vadano ad approssimare il più possibile la forma dell'onda. Ogni campione è formato da un dato numero di bit e viene prelevato con ritmo constante dal suono. Una frequenza di campionamento rappresenta il numero di campioni prelevati in un secondo ed è un numero che di solito oscilla tra gli 8000 ed i 44100 samples al secondo.
\\\\ 
Lo scopo dello script per l'estrazione delle feature è quello di andare a generare tre dataset: 
\begin{itemize}
    \item \textit{data.arff}: Un dataset .arff, che verrà utilizzato per la calssificazione, che conterrà tutte le feature estratte per ogni singolo audio splittato, organizzando le istanze per la classificazione tramite multiple instance learning come descritto nel paragrafo [\ref{par:dataset}]. In questo dataset l'attributo \textbf{class} assume uno dei valori, "trusted", "broadcast\_intent", "shared\_preferences" o "external\_storage". Per estrarre questi attributi si va a leggere contenuto del file di accompagnamento "desctiption.txt" all'interno di ogni set Acid. Nel caso si sta analizzando un file audio generato da un apk trusted l'attributo verrà semplicemente impostato come "trusted".
    \begin{lstlisting}[language=Python, caption=Get android resources, label = lst:getREs]
     with open(f'{acidDatasetFolder}\\{setFold}\\{"description.txt"}', 'r') as reader:
                                    malwareType = reader.read().split(":")[1]
                                    classe = malwareType
    \end{lstlisting}
    \item \textit{dataBinary.arff}: Questo dataset, utilizzato per la calssificazione, conterrà tutte le feature estratte, ancora una volta, per ogni singolo audio splittato, organizzando i dati in bag. La differenza sta nell'andare ad inserire come attributi della \textbf{class} solamente i valori, "trusted", "malware".
    \item \textit{data.csv}: Il dataset.csv comprenderà solamente le applicazioni provenienti dal dataset "Acid". Le tre colonne che compongono la tabella sono riempite nel seguente modo. Nella prima andranno i nome delle bag\_id, nella seconda, le features delle varie istanze della bag ed infine nella terza colonna l'attributo \textbf{class} che può assume uno dei valori, "trusted", "broadcast\_intent", "shared\_preferences" o "external\_storage". Questo dataset sarà utilizzato per generare un quarto dataset "dataGetPut.arff" come vedremo nel paragrafo \ref{subsub:quartodata}. 
\end{itemize}

Lo script realizzato per estrarre le features, inizializzerà da prima i file dataset di cui sopra, che saranno poi utilizzati per organizzare le features. Lo script provvederà ad iterare nelle cartelle "Acid\_Splitted" e "Trusted\_Splitted" il cui contenuto sono le subdirectory con gli audio splittati generati dallo script precedente [\ref{par:gen}]. Per ogni audio contenuto nelle subdirectory si procede quindi all'estrazione delle features con l'ausilio della libreria "librosa"\cite{libosa}.
Attraverso:
\begin{lstlisting}[language=Python]
 y, sr = librosa.load(splittedAudioPath, mono=True, duration=30)
\end{lstlisting}
Si va a caricare il file audio in formato mono. la variabile "y" rappresenta la serie temporale dell'audio, dunque y[t] corrisponde all'ampiezza della forma dell'onda al campione t-esimo. Mentre la variabile "sr" rappresenta la frequenza di campionamento di "y". 
Successivamente attraverso le istruzioni: 
\begin{lstlisting}[language=Python]
            chroma_stft = librosa.feature.chroma_stft(y=y, sr=sr)
            spec_cent = librosa.feature.spectral_centroid(y=y, sr=sr)
            spec_bw = librosa.feature.spectral_bandwidth(y=y, sr=sr)
            rolloff = librosa.feature.spectral_rolloff(y=y, sr=sr)
            zcr = librosa.feature.zero_crossing_rate(y)
            mfcc = librosa.feature.mfcc(y=y, sr=sr)
\end{lstlisting}
\begin{itemize}
\item chroma\_stft: Calcola a partire dall'onda un cromogramma e ritorna un valore normalizzato di ogni frame. 
\item spectral\_centroid: Calcola il centroide spettrale. L'intensità di un suono in funzione del tempo e della frequenza può essere rappresentato graficamente da uno spettrogramma. Ogni fotogramma di uno spettrogramma viene normalizzato e ne viene estratta la media(centroide).
\item spectral\_bandwidth: Calcola la frequenza della larghezza di banda di ogni frame.
\item spectral\_rolloff: Calcola per ogni frame la frequenza di rolloff.
\item zero\_crossing\_rate: Calcola i passaggi per lo zero in una serie temporale.
\item mfcc: Genera una sequenza di coefficienti cefalici mfcc da una serie temporale.
\end{itemize}
Per ognuno di questi valori è stata estratta la media, ed il risultato inserito nella stringa che poi comporrà una riga del dataset. All'interno del codice sono inseriti diversi controlli per far si che la stringa rispetti la formattazione dei file di tipo ARFF. Infine la stringa, composta dal nome dell'audio (integro), dalle feature di tutti i singoli file audio (splittati) e dalla classe d'appartenenza, viene scritta in una nuova riga del file dataset corrispondente. 

\begin{lstlisting}[language=Python, caption=Arff formatting string, label = lst:splitFun]
                .............
            if (featureOfBag == 1):
                featureOfBag += 1
                to_append_arff = f'{audioTitle}{".wav"}{","}\"{np.mean(chroma_stft)}{","}{np.mean(spec_cent)}{","}{np.mean(spec_bw)}{","}{np.mean(rolloff)}{","}{np.mean(zcr)}{","}'
            else:
                to_append_arff += f'\\n{np.mean(chroma_stft)}{","}{np.mean(spec_cent)}{","}{np.mean(spec_bw)}{","}{np.mean(rolloff)}{","}{np.mean(zcr)}{","}'

            indexMfcc = 1
            for e in mfcc:
                if (indexMfcc == len(mfcc)): 
                    to_append_arff += f'{np.mean(e)}' 
                    to_append += f' {np.mean(e)}'
                else:
                    to_append_arff += f'{np.mean(e)}{","}' 
                    to_append += f' {np.mean(e)}{","}'
                indexMfcc += 1
            if splittedAudio != latestSplitter:
                to_append += "\\n"
                
                .............
                
            fileArff = open('results\\data.arff', 'a', newline='')
            fileArff.writelines(to_append_arff)
            fileArff.close()
\end{lstlisting}

Ogni dataset così generato è composto da 682 istanze che individuano le varie bag, con annesse istanze. Ogni bag corrisponde ad una applicazione. 

\subsubsection{Creazione del quarto dataset}
\label{subsub:quartodata}
Per generare il quarto dataset \textbf{dataGetPut.arff} abbiamo realizzato uno script che lavora su un dataset "smistamento.csv" contenente lo smistamento delle applicazioni. Ovvero in questo dataset sono elencati i nomi di ogni applicativo appartenete al dataset "Acid" e per ognuno e specificato se l'applicativo è di tipo "PUT" o "GET".  
\\Lo script quindi va a creare un nuovo file ARFF in cui verranno inserite le feature \textit{dataGetPut.arff}. Questo dataset sarà utilizzato per la classificazione. Di seguito dato in input il dataset "data.csv", creato attraverso lo script precedente [\ref{par:feat}], va a controllare per ogni istanza del dataset (data.csv) il tipo di classe di appartenenza nel secondo dataset (smistamento.csv) attraverso la ricerca di un uguaglianza del nome dell'applicativo. Infine va a scrivere nel datasetGetPut.arff creato le istanze, le features e la tipologia rilevata. 

\begin{lstlisting}[language=Python, caption=Application Get or Put control, label = lst:splitFun]
    for index, row in dataset_csv.iterrows():  
    typeWav = row[2]  
    nameWav = row[0]
    if typeWav == "trusted":
        continue
    else:
        type = str(getTypeOfApp(str(row[0]).split(".")[0])).replace(" ", "_")
        stringa = f'{nameWav}{","}\"{row[1]}\"{","}{type}\n'

        fileArff = open("results\\dataGetPut.arff", 'a', newline='')
        fileArff.writelines(stringa)
        fileArff.close()
\end{lstlisting}

\section{Il multiple instance learning}
In questo paragrafo faremo una breve panoramica sul \hyperref[subsub:mil]{multiple instance learning} prima però descriveremo  cos'è il  \hyperref[subsub:ml]{machine learning}. 
\subsubsection{Machine learning}
\label{subsub:ml}
Il machine learning è quella branca dell'intelligenza artificiale che si occupa di individuare schemi nei dati al fine di addestrare dei modelli per eseguire poi delle previsioni attraverso l'utilizzo di nuovi dati. I dati sono proprio il punto centrale di questa tecnologia, in quanto proprio il progressivo aumento di disponibilità degli stessi ha reso necessario lo sviluppo di algoritmi che fossero in grado di catturarne la conoscenza. I tipi di machine learning sono tre: 
\begin{enumerate}
    \item \textbf{Apprendimento supervisionato}: Questa tipologia si basa sull'addestramento di un modello, all'algoritmo viene dato in input il dataset da analizzare che comprende l'output atteso. I compiti possono dividersi in due sottocategorie, classificazione e regressione. 
    \begin{itemize}
        \item \textit{Classificazione}: L'obiettivo è, sulla base di osservazioni svolte in precedenza, di andare ad effettuare una predizione delle label di classi di nuove istanze. Le label sono un particolare valore, discreto, che rappresenta in qualche modo un gruppo di dati. Dunque il ruolo svolto dalla classificazione è quello di discriminare l'appartenenza di un set di dati ad una delle classi. A sua volta la classificazione può essere suddivisa in altre sottocategorie. Nel caso in cui le classi su cui effettuare una predizione è composta da due label, si dice che che stiamo effettuando una \textit{classificazione binaria}, altrimenti se le classi sono più di due ci troviamo di fronte ad una \textit{classificazione multiclasse}. 
        
        \item \textit{Regressione}: In questo caso sia l'input che l'output sono continui, di fatti dato in input un dataset, l'output sarà dato da un valore continuo. Ad esempio attraverso la regressione lineare data una variabile predittiva x ed una variabile risposta y, si va a calcolare la retta che va a minimizzare la distanza fra i punti e la retta\cite{pyML}.
    \end{itemize}
    \item \textbf{Apprendimento non supervisionato}: In questa categoria non sono note le classi del training set, la struttura dei dati può essere ignota, per l'estrazione delle informazioni dunque non si utilizza una variabile nota ne una funziona di ricompensa come vedremo nella prossima categoria. Un esempio è il \textit{clustering} in cui si vanno a ricercare dei sottogruppi che individuino relazioni tra i dati. 
    \item \textbf{Apprendimento con rinforzo}: In questa categoria, l'algoritmo di apprendimento sviluppa un sistema che migliori le prestazioni andando ad imparare dagli errori che ha effettuato in precedenza. Quindi il focus è di andare a massimizzare la ricompensa attraverso un approccio di tipo esplorativo. 
    
\end{enumerate}
Come si è potuto notare tutti gli approcci necessitano di un "ingrediente fondamentale" i dati. Uno dei passi primari è il pre-procssing dei dati come ad esempio la normalizzazione. 
Un ulteriore passaggio che va effettuato sul dataset, per la maggior parte degli algoritmi è la suddivisone di quest'ultimo in un \textit{training set} ed un \textit{test set}. Il primo viene utilizzato per la fase di addestramento del modello predittivo, il secondo per valutare la capacità di quanto il modello sia in grado di effettuare una predizione utilizzando nuovi dati, viene detto testing. 

\subsubsection{Multiple instance learning}
\label{subsub:mil}
Il Multiple instance learning rientra nell'apprendimento supervisionato. L'utilizzo di questa metodologia ha come caratteristica principale quella che l'organizzazione delle istanze avviene tramite particolari set che vengono chiamati \textit{bag}. Ad ognuna di queste bag viene associata una label, dunque un'etichetta riguarda tutte le istanze della bag\cite{enwiki:972908596}. Dunque ad un algoritmo di apprendimento automatico vengono date in input più bags, ognuna delle quali è composta da più istanze. 
L'obiettivo del MIL\footnote{Abbreviazione di multiple instance learning} è quello di prevedere la label di nuove bag non utilizzate nella fase di addestramento del modello. Lo sviluppo di questa metodologia si sta ampliando in quanto sono aumentati i dati disponibili e classificarli singolarmente, come avviene in un problema di classificazione standard richiede uno sforzo maggiore. Attraverso il MIL si allevia questo onere\cite{Carbonneau_2018} organizzando appunto i dati in bag contenenti le istanze ed assegnandoli un label. Un apprendimento tramite MIL consente di affrontare sia problemi di classificazione che di regressione. 
\\In questo lavoro le istanze che vanno a comporre una bag sono tutte le feature estratte da ogni audio splittato. Questo significa che ogni bag rappresenta una singola applicazione ed il suo contenuto è composto dall'insieme delle feature, ognuna estratta dal relativo file audio splittato dall'audio integro frutto della conversione dell'apk in .wav. La classe di una bag è invece data dal tipo di applicazione, se trusted o malware come specificato al paragrafo \ref{par:feat}

\section{K-fold validation}
\label{par:kfold}
La k-fold validation è una tecnica che va a dividere il dataset in k sezioni, utilizzando k-1 sezioni per il training e la k-ma sezione per il testing. La procedura viene poi ripetuta andando ad utilizzare una nuova sezione delle k calcolate precedentemente per il testing e le restanti k-1 per il training. Il tutto si ripete per k volte andando di volta in volta ad utilizzare un segmento differente per la fase di testing. Infine tra tutte le osservazioni si va a prendere la media. In WEKA l'algoritmo viene chiamato una k+1 esima volta sull'intero testi di dati\cite{kcross}. Questa tecnica è utile per dataset non molto popolati.  
\begin{figure}[h]
   \begin{minipage}{0.48\textwidth}
     \centering
     \includegraphics[width=0.60\linewidth]{imgs/capitolo4/kfold1.png}
     \caption{k th selected segment}
         \label{fig:cross}
   \end{minipage}\hfill
   \begin{minipage}{0.48\textwidth}
     \centering
     \includegraphics[width=0.60\linewidth]{imgs/capitolo4/weka_kcross.png}
     \caption{K-fold in weka}
     \label{fig:wekacross}
   \end{minipage}
\end{figure}

\section{Precision e Recall}
\label{par:precisionRecall}
Per valutare la bontà di una predizione da parte di un modello di machine learning si utilizzano varie metriche. Tra le più utilizzate troviamo al precision e la recall. Distinguiamo in un processo predittivo i falsi positivi, ovvero quelle istanze etichettate come appartenenti ad una classe ma che in realtà non lo sono ed i veri positivi, ovvero quelle istanze etichettate come appartenenti ad un classe e che lo sono realmente.
\begin{itemize}
    \item \textbf{Precision}: è data dal rapporto tra il numero di veri positivi ed il numero totale di elementi etichettati come appratenti ad una classe, dunque la somma di veri e falsi positivi. Un modello preciso genera pochi falsi positivi, perché quando prevede l'appartenenza ad una classe raramente sbaglia, mentre i falsi negativi potrebbero essere molti non prevedendo quindi tutte le istanze che dovrebbero appartenere ad una classe. Dunque la precision peggiora se vi sono tanti falsi positivi. 
    \item \textbf{Recall}: Questa metrica è data dal rapporto tra il numero di veri positivi ed il numero di istanze che realmente vi appartengono. Un alto valore di recall indica che il modello recupera tutte le istanze che appartengono alla classe. Tuttavia potrebbero esserci molti falsi positivi. Dunque la recall peggiora se vi sono tanti falsi negativi. 
\end{itemize}
Spesso accade che al migliorare della precisione si assiste ad un peggioramento della recall e viceversa. Bisogna dunque trovare un modello che cerchi un giusto compromesso. 
	
\chapter{Sperimentazioni}
\label{chap:sperim}
In questo capitolo verranno riportati i risultati osservati nella fase di classificazione dei dataset ottenuti dalle features dai file audio splittati, ottenuti a partire dalle applicazioni Andorid. Da prima vedremo come sono composti i dataset ed in seguito quali sono i risultati ottenuti dalle classificazioni. Il processo di addestramento di modelli di multiple instance learning è avvenuto attraverso l'utilizzo del software WEKA [\ref{par:weka}]. 
La bontà delle classificazioni effettuate si è calcolata dalle due metriche precision e recall sopra descritte [\ref{par:precisionRecall}]. Nel processo di training di ogni modello di classificazione utilizato si è impiegato l'utilizzo della K-Fold Validation [\ref{par:kfold}], nel dettaglio la divisione e ripetizione è avvenuta sul 10\% del dataset, impostando il valore $K = 10$, di conseguenza i valori di precision e recall rappresentano la media aritmetica delle 10 iterazioni eseguite durante l'addestramento.
\section{Classificazione multi classe}
Come detto in precedenza, al paragrafo \ref{par:datasetApk}, le applicazioni utilizzate provengono da due dataset "trusted" e "Acid". Le applicazioni Acid dono di diverso tipo, ”broadcast\_intent”, ”shared\_preferences” o ”external\_storage”. Dopo aver effettuato una prima conversione delle applicazioni Android in file audio di tipo wav, lo splitting degli stessi è stato effettuato in due passaggi. Nella prima operazione di splitting, la suddivisione è stata effettuata in intervalli di circa  35 minuti (2092 secondi) ricavando quindi dall'estrazione delle feature da ogni file audio suddiviso, un dataset "\textit{data\_2092.arff}" di 682 bag, il maggior numero delle quali composte da circa 8 istanze. Nella seconda operazione di splitting, la suddivisione è stata effettuata in intervalli di circa 17 minuti (1046 secondi) in questo modo il numero di bag che vanno a comporre il dataset "\textit{data\_1046.arff}" è come prima di 682 bag, la maggior parte delle quali popolata però da 16 istanze.
In figura \ref{fig:multiclassdataset} si può osservare la suddivisione delle classi. 
\begin{figure}[h]
\centering
    \includegraphics[width=0.9\linewidth]{imgs/capitolo5/multicalsse/682 bag 4 classi.png} 
    \caption{Multi class dataset}
    \label{fig:multiclassdataset}
\end{figure}
\FloatBarrier
La classificazione è stata eseguita adottando il criterio della K-Fold validation [\ref{par:kfold}].
Nello specifico l'algoritmo utilizzato è stato il TLC che ha restituito per il dataset "\textbf{data\_2092.arff}":
\begin{itemize}
    \item Precision: 0.847 
    \item Recall: 0.845
\end{itemize}
mentre per il dataset "\textbf{data\_1046.arff}" i valori osservati sono:
\begin{itemize}
    \item Precision: 0.844 
    \item Recall: 0.845
\end{itemize}

In figura \ref{fig:plotMulticlasse} si può osservare come le istanze del test set sono state classificate. Nello specifico sulle ascisse è riportata la classe predetta dal modello mentre sulle ordinate la classe di appartenenza originale. Alle istanze è stata applicata Jitter per una migliore visualizzazione. Le istanze rappresentate da una 'x' sono state predette correttamente, ovvero la classe coincide con la classe predetta dal modello, mentre le classi rappresentate dal quadratino sono quelle istanze erroneamente classificate dal modello, la cui classe d'appartenenza è la corrispondete sull'asse delle ordinate che da anche il colore alle istanze, mentre la predetta è la classe corrispondete sulle ascisse. 
\begin{figure}[h]
\centering
    \includegraphics[width=0.9\linewidth]{imgs/capitolo5/multicalsse/data_multiclass_visual.png} 
    \caption{Multi class classification}
    \label{fig:plotMulticlasse}
\end{figure}
\FloatBarrier

\section{Classificazione binaria Trusted - Malware}
La seconda sperimentazione effettuata riguarda la classificazione dei dataset le cui classi d'appartenenza sono "trusted" e "malware". I dataset popolati da 682 bag sono stati ricavati da istanze di audio splittati in circa  35 minuti (2092 secondi) da cui si è ottenuto il dataset "dataBinary\_2092.arff" e da audio suddivisi in intervalli di circa 17 minuti (1046 secondi) da cui abbiamo ricavato il dataset "dataBinary\_1046.arff". Le classi delle bag sono composte da 200 bag con label "trusted" e 482 bag con label "malware" come si può osservare dall'istogramma in figura \ref{fig:bindataset}.
\begin{figure}[h]
\centering
    \includegraphics[width=0.9\linewidth]{imgs/capitolo5/multicalsse/binaryclass.png} 
    \caption{Binary class dataset}
    \label{fig:bindataset}
\end{figure}
\FloatBarrier

Le classificazioni hanno riguardato diversi algoritmi, i risultati, valutati attraverso i valori di precision e recall sono stati: 

\begin{center}
 \begin{tabularx}{0.8\textwidth} { | >{\raggedright\arraybackslash}X | >{\centering\arraybackslash}X | >{\raggedleft\arraybackslash}X | }
        \hline
        \multicolumn{3}{|c|}{\textbf{Dataset dataBinary\_2092}}\\
        \hline
       \textbf{ALG} & \textbf{PRECISION} & \textbf{RECALL} \\
       \hline
       MIEMDD  & 0.999  & 0.999  \\
        \hline
       MIDD   & 0.997  & 0.997  \\
        \hline
       QUICK DD  & 0.996  & 0.996  \\
        \hline
       TLC  & 0.993  & 0.993  \\
        \hline
       MITI  & 0.981  & 0.981  \\
        \hline
       MIRI  & 0.980  & 0.979  \\
        \hline
       MILR  & 0.949  & 0.946  \\
        \hline
       CitationKNN  & 0.924  & 0.915  \\
       \hline
       MDD  & 0.852 & 0.818  \\
        \hline
       MISVM  & 0.806  & 0.733  \\
        \hline
       TLD  & 0.79  & 0.299  \\
       \hline
    \end{tabularx}
    \end{center}
\begin{center}
    
    %%secodna tabella%%
 \begin{tabularx}{0.8\textwidth} { | >{\raggedright\arraybackslash}X | >{\centering\arraybackslash}X | >{\raggedleft\arraybackslash}X | }
        \hline
        \multicolumn{3}{|c|}{\textbf{Dataset dataBinary\_1046}}\\
        \hline
       \textbf{ALG} & \textbf{PRECISION} & \textbf{RECALL} \\
       \hline
       MIEMDD  & 0.997  & 0.997  \\
        \hline
             TLC  & 0.996  & 0.996  \\
        \hline
         QUICK DD  & 0.991  & 0.991  \\
        \hline
       MITI  & 0.913  & 0.902  \\
        \hline
       MIRI  & 0.913  & 0.902  \\
        \hline
       MILR  & 0.981  & 0.981  \\
        \hline
       CitationKNN  & 0.916  & 0.905  \\
       \hline
       MDD  & 0.915 & 0.905  \\
        \hline
       MISVM  & 0.802  & 0.724  \\
        \hline
        \end{tabularx}
\end{center}

Dunque il risultato migliore si è ottenuto con l'algoritmo MIEMDD in entrambi i dataset. 

\section{Classificazione binaria Apk\_Get - Apk\_Put}
In questa sperimentazione, i dataset comprendono solo le applicazioni affette da malware provenienti dal dataset "Acid". Le bag di ogni dataset sono quindi in totale 482, suddivise in due classi Apk\_Get e Apk\_Put come descritto nel paragrafo \ref{par:dataset}. 
Nella figura \ref{fig:getputclassdataset} possiamo osservare l'istogramma relativo alle due classi sopra descritte. 
\begin{figure}[h]
\centering
    \includegraphics[width=0.9\linewidth]{imgs/capitolo5/multicalsse/getputclass.png} 
    \caption{Get - Put class dataset}
    \label{fig:getputclassdataset}
\end{figure}
\FloatBarrier

Anche in questo caso la classificazione ha ricoperto l'utilizzo di più algoritmi. La classificazione è stata eseguita sui due dataset "dataGetPut\_2092.arff" e "dataGetPut\_1046.arff" e la valutazione eseguita sulla bontà dei risultati di precision e recall. I valori osservati sono stati i seguenti: 
\begin{center}
 \begin{tabularx}{0.8\textwidth} { | >{\raggedright\arraybackslash}X | >{\centering\arraybackslash}X | >{\raggedleft\arraybackslash}X | }
        \hline
        \multicolumn{3}{|c|}{\textbf{Dataset dataGetPut\_2092}}\\
        \hline
       \textbf{ALG} & \textbf{PRECISION} & \textbf{RECALL} \\
       \hline
       TLC  & 0.979  & 0.979  \\
        \hline
       MITI   & 0.971  & 0.971  \\
        \hline
       MIRI  & 0.965  & 0.965  \\
        \hline
       MIDD  & 0.878  & 0.861  \\
        \hline
       MDD  & 0.804  & 0.712  \\
        \hline
       TLDSIMPLE  & 0.755  & 0.519  \\
        \hline
       MIEMDD  & 0.751  & 0.502  \\
        \hline
       QUICKDD  & 0.729  & 0.726  \\
       \hline
       MILR  & 0.599 & 0.562  \\
        \hline
       BOOST  & 0.498  & 0.498  \\
        \hline
       WRAPPER  & 0.498  & 0.498  \\
       \hline
        CitationKNN  & 0.497  & 0.498  \\
       \hline
        SIMPLEMI  & 0.494  & 0.498  \\
       \hline
    \end{tabularx}
    \end{center}
\begin{center}
    
    %%secodna tabella%%
 \begin{tabularx}{0.8\textwidth} { | >{\raggedright\arraybackslash}X | >{\centering\arraybackslash}X | >{\raggedleft\arraybackslash}X | }
        \hline
        \multicolumn{3}{|c|}{\textbf{Dataset dataGetPut\_1046}}\\
        \hline
         \textbf{ALG} & \textbf{PRECISION} & \textbf{RECALL} \\
       \hline
       TLC  & 0.992  & 0.992  \\
        \hline
       MITI   & 0.979  & 0.979  \\
        \hline
       MIRI  & 0.975  & 0.975  \\
        \hline
       MIDD  & 0.913  & 0.913  \\
        \hline
       MDD  & 0.805  & 0.703  \\
        \hline
       TLDSIMPLE  & 0.754  & 0.515  \\
        \hline
       MIEMDD  & 0.700  & 0.618  \\
        \hline
       QUICKDD  & 0.869  & 0.838  \\
       \hline
       MILR  & 0.574 & 0.573  \\
        \hline
       BOOST  & 0.498  & 0.498  \\
        \hline
       WRAPPER  & 0.498  & 0.498  \\
       \hline
        CitationKNN  & 0.497  & 0.498  \\
       \hline
        SIMPLEMI  & 0.494  & 0.498  \\
       \hline
        \end{tabularx}
\end{center}
Nel classificare questi dataset il risultato migliore è stato ottenuto con l'utilizzo dell'algoritmo TLC. 
	% Inserite qui gli altri capitoli:
	%\input{miocapitolo1}
	%\input{miocapitolo2}
	%%%%%%%%%%%%%%%%%%%%%%%%%%%%%%%%%%%%%%%%%%%%%%%%%%%%%%%%%%%

\chapter{Conclusioni}
\label{chap:Conclusioni}

\begin{flushright}
\textit{Ing.Francesco Mercaldo, Dott.ssa Rosangela Casolare , Andrea D'Aguanno} 
\end{flushright}
	

	% === Bibliografia ====================================
	\newpage
	\bibliographystyle{IEEEtran}
	\bibliography{bibliografia-tesi.bib}

\end{document}

%%%%%%%%%%%%%%%%%%%%%%%%%%%%%%%%%%%%%%%%%%%%%%%%%%%%%%%%%%%
